\chapter{Axioms}\label{chap:axioms}

We now give a full account of all the axioms / unproven results used
throughout the project.
We can differentiate these axioms in the following categories:

\begin{itemize}
\item \textbf{Generated code instances}: axioms that are needed for the generated code to be well
  behaved.
\item \textbf{Uninterpreted functions}: axioms that pose the existence of a function without
  providing an implementation.
\item \textbf{Pre-theorems}: axioms that should be turned into theorems.
\item \textbf{Theorem sorries}: theorems with absent proofs that should hold.
\item \textbf{Subgoal sorries}: subogals that have been sorried-out but are not the most
  critical part of the theorems they belong to.
\end{itemize}

\section{Generated code instances}

There are two instances that we have declared without providing any proof due to
time constraints. For the \texttt{SortK} we have declared instances of the
\texttt{DecidableEq} and \texttt{LawfulBEq} classes without providing any
proofs. This has been done due to time constraints and code stability. However, these should be provable.

\texttt{SortK} is basically a sequence of elemts of sort \texttt{SortKItem},
which in turn has injections from every sort. This means that every time a new
type was generated (by including more parts of the KEVM semantics in the
generated code by adding more summaries) the proof should have to be adjusted.

The main application of these instances is to prove the results presents in
\href{https://runtimeverification.github.io/evm-equivalence/docs/EvmEquivalence/Interfaces/GasInterface.html}{\texttt{GasInterface.lean}}.

\section{Uninterpreted functions}

Some functions in K are left for the backends to implement. These are called
``hooked functions''. What this means is that these functions don't have
definition in the compiled K files. Instead, when such a function is
encountered, K will use the implementation that the executing backend provides
(the hook).

In our context, hooked functions come in two flavours: implemented or
unimplemented. Implemented hooked functions are present in the
\href{https://runtimeverification.github.io/evm-equivalence/docs/EvmEquivalence/KEVM2Lean/Prelude.html}{\texttt{Prelude.lean}},
and unimplemented or uninterpreted hooked functions are declared as axioms in
the
\href{https://runtimeverification.github.io/evm-equivalence/docs/EvmEquivalence/KEVM2Lean/Func.html}{\texttt{Func.lean}}
file.

All axioms in \texttt{Func.lean} should only be unimplemented hooked functions.

\section{Pre-theorems and theorem sorries}

All axioms that are not uninterpreted functions should become a theorem at some
point. We also have theorems that no proof has been attempted. All such
statements are thought to hold, and not holding would imply discrepancies of
some sort between the models.

\section{Subgoal sorries}

Some theorems have some subgoals sorried out. This is because we have
provided the general structure of the proof, but haven't got the time yet to
close all subgoals.

All such subgoals should be provable requiring at most some small tweaking of
the surrounding functions.

\section{Gas subgoal sorries}

One special (almost systematic) kind of subgoal sorry is that of gas-related subgoals.
For the current opcodes targeted for equivalence we have not prioritized (given the time constraints) proving the equivalence of gas consumption or other gas-related obligations.

The reason for this is that we consider gas agreeance as less critical than the other proof obligations that are covered.
In fact, if the EvmYul model allowed for a gasless mode of execution, as does KEVM, it is very probable that the first version of the equivalence proofs would have made use of it.

Another reason for not prioritizing gas-related goals is that they are often quite time consuming. A good direction of research is to apply more automatic reasoning methods such as SMT-based tactics or LLMs.
