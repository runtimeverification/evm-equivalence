\chapter{Equivalences}

With what we have defined so far we are in a position to start proving
equivalence between KEVM and the EvmYul model. However, first, we need to
pinpoint exactly what do we mean by equivalence.

\section{Defining Equivalence}

We define now the notion of equivalence we're dealing with, and to what extent
is it formalized and proven.

The overarching goal is to prove that there exists a bisimulation relation between
KEVM and the EvmYul model.

We start by refreshing the definition of a bisimulation.

\begin{definition}[Bisimulation relation]\label{def:bisimulation}
Given a transition system $(S, \rightarrow)$, a binary relation $R$ on $S$ is a
bisimulation if and only if for every pair of states $(p, q)\in R$ we have:
\begin{itemize}
\item if $p \rightarrow p'$ then there is $q \rightarrow q'$ with $(p', q') \in
  R$;
\item if $q \rightarrow q'$ then there is $p \rightarrow p'$ with $(p', q') \in
  R$.
\end{itemize}
\end{definition}

In our context,
\begin{itemize}
\item $S$ is the union of both KEVM and EvmYul states ($S_{\text{KEVM}}$ and
  $S_{\text{EvmYul}}$ respectively). So we have $S_{\text{KEVM}}\, \bigcup\, S_{\text{EvmYul}} = S$ and $S_{\text{KEVM}}\, \bigcap\, S_{\text{EvmYul}} = \emptyset$.
\item The transition relation $\rightarrow$ is execution in any of the models. In particular, we have that $\rightarrow\,
\subset S_{\text{KEVM}} \times S_{\text{KEVM}} \,\bigcup\, S_{\text{EvmYul}} \times S_{\text{EvmYul}}$.
\item $R = \bigl\{\bigl( p, \text{stateMap}(p) \bigr)\, |\, p\in S_{\text{KEVM}}\bigr\} \,\bigcup\, \bigl\{\bigl(\text{stateMap}'(q),
  q\bigr)\, |\, q\in S_{\text{EvmYul}}\bigr\}$ where $\text{stateMap}$ is
  defined in \ref{def:stateMap} and $\text{stateMap}'\,:\,
  S_{\text{EvmYul}}\rightarrow S_{\text{KEVM}}$ has not yet been defined.

\end{itemize}
Note that $R \subsetneq S_{\text{KEVM}}\times S_{\text{EvmYul}}$.
With this in mind, we currently have machinnery to prove the first implication
of the bisimulation definition.

Given KEVM states $p, p'$ with $p \rightarrow p'$, we know that
$\bigl( p, \text{stateMap}(p) \bigr)$ and $\bigl( p', \text{stateMap}(p')
\bigr)$ belong to $R$. Hence, to prove the first item we only need to show that
$\text{stateMap}(p) \rightarrow \text{stateMap}(p')$.

\subsubsection{Clarifications on $\rightarrow$}

Saying ``the transition relation $\rightarrow$ is execution in any of the
models'' is admittedly quite loose. So we can make its definition more precise
as follows.

\begin{definition}[Transition function $\rightarrow$]
Given $p, p' \in S$, we have that $p\rightarrow p'$ if
\begin{itemize}
\item $p, p'\in S_{\text{KEVM}}$ and \texttt{Rewrites} $p$ $p'$ holds.
\item $p, p'\in S_{\text{EvmYul}}$ and executing the
  \texttt{step}[\ref{def:EVM.step}] or \texttt{X}[\ref{def:EVM.X}] function on $p$ yields $p'$.
\end{itemize}
\end{definition}

Now we can start describing the proof process more clearly.

\section{Proof Outline}

Given all the above and $f\in\{\texttt{step}, \texttt{X}\}$, the first
implication translates to

\begin{equation}\label{eq:bisimulation-explicit1}
\forall p, p'\in S_{\text{KEVM}}\,\, \texttt{Rewrites}\,\, p \,\, p'
\Rightarrow f\bigl(\text{stateMap}(p)\bigr) = \text{stateMap}(p').
\end{equation}

While the second translates to
\begin{equation}\label{eq:bisimulation-explicit2}
\forall q, q'\in S_{\text{EvmYul}}\,\, f(q) = q' \Rightarrow
\texttt{Rewrites}\,\, \text{stateMap}'(q)\,\, \text{stateMap}'(q').
\end{equation}

The above statement is still not fully precise, since we're only feeding
elements of $S_{\text{EvmYul}}$ to $f$, but both \texttt{step} and \texttt{X}
take more arguments. However, the only argument that we'll fix is the opcode
being executed.

This means that we will be reproducing the above argument on a per-opcode basis.
The reasoning being that showing equivalence at the opcode-execution level is
fundamental enough to argue that both models share the same understanding of the
semantics.

The following example should shine enough light on the
specificities that we have left out so far.

\section{Case Study: Stack Operations}

In this example we will be covering proofs for \ref{eq:bisimulation-explicit1}
for ``stack operation opcodes'', meaning opcodes that perform operations on the
stack such as \texttt{ADD} or \texttt{SHL}. Since we're reproducing the same
argument for each opcode, the proof process is architecturally similar across
the files.

In fact, for some opcodes, we can bundle them together and reproduce the above
argument at once per opcode class.

\subsection{Opcode Bundling}

The main criteria for creating these opcode classes is based on their KEVM
summaries. To understand how KEVM summaries condition which opcodes go to each
class (which is still a somewhat arbitrary taxonomy) we assume understanding of
the
\href{https://github.com/runtimeverification/evm-equivalence/tree/master/EvmEquivalence/Equivalence#anatomy-of-a-rewrite-rule}{anatomy
  of rewrite rules}.

There are two main frictions when bundling opcodes together: how fine-grained we
need the state representation to be and how many ``steps'' does it take for KEVM
to encode the state changes.

The fine-grainedness of the state refers to how abstracted the satate is. For
instance, instead of specifying symbolic values for all the components of the
\texttt{network} cell, we just have a symbolic \texttt{network} cell variable.
This is not troublesome in general.

What is more defining for opcode bundling is the amount of steps or conditions
needed to encode the state changes performed per opcode. For example, for the
\texttt{ADD} opcode we need two conditions (one per operation) to define the
value that will be put on the top of the stack. The first condition adds the
first two items of the stack, and the second one takes the corresponding
modulus:
\begin{verbatim}
(defn_Val3 : «_+Int_» W0 W1 = some _Val3)
(defn_Val4 : chop _Val3 = some _Val4)
\end{verbatim}
In this case, \texttt{_Val4} is going to be the value introduced at the top of
the stack.

We can contrast this to the \texttt{DIV} opcode, which only takes one condition
to encode the value of the newly added stack item:
\begin{verbatim}
(defn_Val3 : «_/Word__EVM-TYPES_Int_Int_Int» W0 W1 = some _Val3)
\end{verbatim}

With this in mind, we first define an inductive type representing all opcodes in
this class. The class being ``opcodes that perform stack operations and take one
condition to summarize the new value added to the stack.''

\begin{definition}[stackOps_op]\label{def:OneOp.stackOps}
\lean{OneOpEquivalence.stackOps_op}\leanok
Class of opcodes consisting of
\texttt{DIV}, \texttt{SDIV}, \texttt{MOD}, \texttt{SMOD}, \texttt{SIGNEXT},
\texttt{SLT}, \texttt{SGT}, \texttt{AND}, \texttt{XOR}, \texttt{NOT},
\texttt{BYTE}, \texttt{SHL}, \texttt{SHR}, \texttt{SAR}.
\end{definition}

This definition will allow us to generalize the rest of the description to the
whole class of opcodes.

In the remaining of this chapter we're going to be lying down the formalization
process for \ref{eq:bisimulation-explicit1} for the opcode class
\texttt{stackOps_op}.

\subsection{$\forall p, p'\in S_{\text{KEVM}}\,\, \texttt{Rewrites}\,\, p \,\, p'$}

The first stop is to, given the class of opcodes \texttt{stackOps_op},
identify all $p, p'\in S_{\text{KEVM}}$ such that $\texttt{Rewrites}\,\, p \,\,
p'$ holds by means of executing any of the opcodes in that equivalence class
(written as $\texttt{Rewrites}_{\texttt{stackOps_op}}\,\, p \,\, p'$).
To achieve this we rely on the following two definitions.

\begin{definition}[oneOpLHS]\label{def:oneOpLHS}
\lean{OneOpEquivalence.oneOpLHS}\leanok\uses{def:OneOp.stackOps}

\texttt{oneOpLHS} is a symbolic state representing all KEVM states that
rewrite to some other state by means of executing any opcode of the
\texttt{stackOps_op} class. Viewed as a
function, $$\mathcal{Im}(\texttt{oneOpLHS}) \supset \{p\,\, |\,\, \exists
p'\,\,\texttt{Rewrites}_{\texttt{stackOps_op}}\,\, p \,\, p'\}.$$

\end{definition}

\begin{definition}[oneOpRHS]\label{def:oneOpRHS}
\lean{OneOpEquivalence.oneOpRHS}\leanok\uses{def:OneOp.stackOps}

\texttt{oneOpRHS} is a symbolic state representing all KEVM states that are a
rewrite of some other state by means of executing any opcode of the
\texttt{stackOps_op} class. Viewed as a
function, $$\mathcal{Im}(\texttt{oneOpRHS}) \supset \{p'\,\, |\,\, \exists
p\,\,\texttt{Rewrites}_{\texttt{stackOps_op}}\,\, p \,\, p'\}.$$

\end{definition}

The next step is, given $p\in\mathcal{Im}(\texttt{oneOpLHS})$ and
$p'\in\mathcal{Im}(\texttt{oneOpRHS})$, to find which are the necessary
conditions such that $\texttt{Rewrites}_{\texttt{stackOps_op}}\,\, p \,\, p'$
holds. For this we turn to the constructors of the $\texttt{Rewrites}$ type.

Each constructor represents an opcode, encoding the KEVM summary of executing such
opcode on a symbolic state. The constructors of the rewrite type in turn contain
all the conditions (such as \texttt{defn_Val3} above) to satisfy the rewrite
summary. That is, the necessary conditions that ensure
$\texttt{Rewrites}_{\texttt{stackOps_op}}\,\, p \,\, p'$. The following theorem
proves that for any two such $p, p'$, under the restrictions posed by the
constructors associated to the \texttt{stackOps_op} class, $p\rightarrow p'$.

Note that, \texttt{oneOpLHS} and \texttt{oneOpRHS} being symbolic states, it is
also correct to state \texttt{Rewrites} \texttt{oneOpLHS} \texttt{oneOpRHS}.
However, this means that \texttt{oneOpLHS} rewrites to \texttt{oneOpRHS} for all
possible value of its symbolic parameters. This is generally a false statement
without restricting their symbolic values.

\begin{theorem}[Rewrites oneOpLHS oneOpRHS]\label{thm:oneOpRw}
\lean{OneOpEquivalence.rw_oneOpLHS_oneOpRHS}\leanok
\uses{def:oneOpLHS,def:oneOpRHS,def:Rewrites}

Given the restrictions posed by the \texttt{Rewrites} constructors corresponding
to the \texttt{stackOps_op} opcode class,
$\texttt{Rewrites}_{\texttt{stackOps_op}}$ \texttt{oneOpLHS} \texttt{oneOpRHS}.

\end{theorem}

\begin{proof}
\leanok

We have that \texttt{oneOpLHS} and \texttt{oneOpRHS} are generalizations of the
states represented in the \texttt{Rewrites} constructors for the
\texttt{stackOps_op} class. As such, if we specialize the definitions of
\texttt{oneOpLHS} and \texttt{oneOpRHS} down to the opcode, since we're assuming
the conditions of satisfaction for every \texttt{stackOps_op} opcode summary,
the result follows.

\end{proof}

\subsection{$f\bigl(\text{stateMap}(p)\bigr) = \text{stateMap}(p')$}

Now that we know the conditions under which
$\texttt{Rewrites}_{\texttt{stackOps_op}}$ \texttt{oneOpLHS} \texttt{oneOpRHS} holds,
we have to show $f\bigl(\text{stateMap}(\texttt{oneOpLHS})\bigr) =
\text{stateMap}(\texttt{oneOpRHS})$ for $f\in\{\texttt{step}, \texttt{X}\}$.

But in order to compute $f$, first we need to know the values of
$\text{stateMap}(\texttt{oneOpLHS})$ and $\text{stateMap}(\texttt{oneOpRHS})$.
Similarly as with the EvmYul summaries [\ref{chap:EvmYulSummaries}], we do this
with theorems.

\begin{theorem}[oneOpLHS_equiv]\label{thm:oneOpLHS_equiv}
\lean{OneOpEquivalence.oneOp_prestate_equiv}\leanok\uses{def:stateMap}
This theorem allows for a clean rewrite of $\text{stateMap}(\texttt{oneOpLHS})$.
\end{theorem}
\begin{proof}
\leanok
This result follows by specifying it down to the opcode and executing the
stateMap function.
\end{proof}

\begin{theorem}[oneOpRHS_equiv]\label{thm:oneOpRHS_equiv}
\lean{OneOpEquivalence.oneOp_poststate_equiv}\leanok\uses{def:stateMap}
This theorem allows for a clean rewrite of $\text{stateMap}(\texttt{oneOpRHS})$.
We tipically go beyond just a pure rewrite of the stateMap function and also
simplify some of the initial expressions present either in \texttt{oneOpRHS} or
as a result of the stateMap computation.
\end{theorem}
\begin{proof}
This result follows by specifying it down to the opcode and executing the
stateMap function.
\end{proof}

Equipped with these theorems, we can now state and prove the main theorems for
the functions \texttt{step} and \texttt{X}. Since \texttt{X} iteratively applies
\texttt{step}, we start with the \texttt{step} function.

\begin{theorem}[step_oneOp]\label{thm:step_oneOp}
\lean{OneOpEquivalence.step_oneOp_equiv}\leanok
\uses{def:stateMap,def:EVM.step,def:oneOpLHS,def:oneOpRHS}

Assume

\begin{itemize}
\item All the conditions needed to satify the summaries of
  $\texttt{Rewrites}_{\texttt{stackOps_op}}$,
\item A schedule set to CANCUN,
\item Enough gas per opcode,
\item Well formed gas and program counter parameters in \texttt{oneOpLHS},
\item Non-negative elements in the stack of \texttt{oneOpLHS}.
\end{itemize}

Then, evaluating the function \texttt{EVM.step} with any of the
\texttt{stackOps_op} opcodes on $\text{stateMap}(\texttt{oneOpLHS})$ results in
$\text{stateMap}(\texttt{oneOpRHS})$.

\end{theorem}

\begin{proof}
\uses{thm:oneOpLHS_equiv,thm:oneOpRHS_equiv,thm:cancun_def}

The proof consists of several parts, but many are shared among the
\texttt{stackOps_op} class, such as the gas or program counter proof
obligations.

The main semantic challenge for this class of opcodes is to show that the
operations performed on the stack correspond to each other. So, for example, for
the \texttt{DIV} case, once all the execution-related bureaucracies have been
dealt with, the main goal is the following:

\begin{verbatim}
intMap W0 / intMap W1 = intMap (divWord W0 W1)
\end{verbatim}

Where \texttt{W0} and \texttt{W1} are the first two elements of
\texttt{oneOpLHS}'s stack.

What this proof obligation is stating is that mapping the first two elements of
a KEVM stack to EvmYul and then dividing them, is the same as dividing the two
elements in KEVM and them mapping them to EvmYul.

\end{proof}

\begin{theorem}[X_oneOp]\label{thm:X_oneOp}
\lean{OneOpEquivalence.X_oneOp_equiv}\leanok
\uses{def:stateMap,def:EVM.X,def:oneOpLHS,def:oneOpRHS}

Assume

\begin{itemize}
\item All the conditions needed to satify the summaries of
  $\texttt{Rewrites}_{\texttt{stackOps_op}}$,
\item A schedule set to CANCUN,
\item Enough gas per opcode,
\item Well formed gas parameter in \texttt{oneOpLHS},
\item The program counter of \texttt{oneOpLHS} set to zero,
\item Non-negative elements in the stack of \texttt{oneOpLHS},
\item A well formed stack in \texttt{oneOpLHS}
\item The program to be executed in \texttt{oneOpLHS} be a single-opcode program
  consisting of any opcode of the \texttt{stackOps_op} class.
\end{itemize}

Then, evaluating the function \texttt{EVM.X} with any of the
\texttt{stackOps_op} opcodes on $\text{stateMap}(\texttt{oneOpLHS})$ results in
$\text{stateMap}(\texttt{oneOpRHS})$.

\end{theorem}

\begin{proof}
\uses{thm:oneOpLHS_equiv,thm:oneOpRHS_equiv,thm:cancun_def,thm:X_bad_pc}

Similarly as for the \texttt{EVM.step} function, we use tactics to execute
\texttt{EVM.X} symbolically. The hypothesis space allows us to get rid of all
reverting cases and confront the semantic equivalence similarly as in the proof
for the \texttt{EVM.step} function.

\end{proof}

With these two theorems proven we can conclude that under the conditions of
satisfaction of the summaries for the \texttt{stackOps_op} opcodes, together
with some well-formedness constraints, there is a similarity relationship
(equation \ref{eq:bisimulation-explicit1}) between executing the
\texttt{stackOps_op} opcodes in KEVM and executing them in the EvmYul model.

\section{Future Work}

There are a number of items that remain as future work. For one, formalizing and
proving equation \ref{eq:bisimulation-explicit2} in a similar fashon to what
we've shown with equation \ref{eq:bisimulation-explicit1}.
But within the formalization efforts of equation
\ref{eq:bisimulation-explicit1}, it remains to close a number of proof
obligations and formalizing other classes of opcodes.
