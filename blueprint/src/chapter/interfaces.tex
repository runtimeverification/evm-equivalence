\chapter{Interfaces}

We are describing the files in the
\href{https://github.com/runtimeverification/evm-equivalence/tree/master/EvmEquivalence/Interfaces}{Interfaces}
folder.

We have to reason about the EvmYul model and the generated K code. But we soon
face the challenge that the EvmYul model is more geared towards execution than
reasoning and that the generated K code is particularly idiosyncratic.

This is not to the detriment of any of the models, but we do need reasoning
interfaces for these models to smooth the reasoning process.

\section{EvmYul Interface}

This
\href{https://github.com/runtimeverification/evm-equivalence/blob/master/EvmEquivalence/Interfaces/EvmYulInterface.lean}{interface}
contains results that aid with the reasoning of EvmYul functions.

There are several type of results in this interface:

\begin{itemize}
\item \texttt{USize} results
\item Compatibility axioms
\item Bytes manipulation results
\item \texttt{UInt256} results and conversions
\item Opcode semantics
\end{itemize}

We won't describe the entirety of the results provided in this interface, but
we'll discuss the two most pressing categories: compatibility axioms and opcode semantics.

\subsection{Compatibility Axioms}

Of particular importance are two axioms necessary to the whole operation. The
first one is \texttt{ffi_zeroes}, which involves the EvmYul \texttt{zeroes} function.

\subsubsection{\textbf{Aixiom \texttt{ffi_zeroes}}}

\begin{definition}[zeroes]\label{def:zeroes}\lean{ffi.ByteArray.zeroes}
\texttt{zeroes len} produces a byte array of size \texttt{len} where every byte
is the zero byte.

This part of the EvmYul FFI.
\end{definition}

Crucially, the \texttt{zeroes} function is a
\href{https://en.wikipedia.org/wiki/Foreign_function_interface}{foregin function interface (FFI)}
function. This means that there is no Lean implementation of it. Rather, when
executed, Lean will call en external function to perform the computational
action.
However, the \texttt{zeroes} function plays a crucial role in important
functions such as writing to memory.

That is why, in order to reason about it, we pose an axiom stating the desired
behavior of the \texttt{zeroes} function.

\begin{theorem}[Axiom: ffi_zeroes]\label{ax:ffi_zeroes}\lean{Axioms.ffi_zeroes}\uses{def:zeroes}
\texttt{zeroes len} returns a byte array of size \texttt{len} where every byte
is the zero byte.
\end{theorem}

\subsubsection{\textbf{Aixiom \texttt{toBytesBigEndian_rw}}}

The second needed axiom is to bypass a \texttt{private} attribute of the EvmYul
\texttt{toBytes'} function defined below:

\begin{definition}[toBytes']\label{def:toBytes'}\leanok
\texttt{toBytes' n} converts a given natural number \texttt{n} to its little
endian bytes represenation.

Crucially, this definition is \texttt{private}.
\end{definition}

The \texttt{toBytes'} function is used by \texttt{toBytesBigEndian}:

\begin{definition}[toBytesBigEndian]
\label{def:toBytesBigEndian}\uses{def:toBytes'}\lean{EvmYul.toBytesBigEndian}\leanok
\texttt{toBytes' n} converts a given natural number \texttt{n} to its big
endian bytes represenation.

\texttt{toBytesBigEndian = List.reverse ∘ toBytes'✝}
\end{definition}

When dealing with \texttt{toBytesBigEndian} we need to also deal with
\texttt{toBytes'}. The following axiom states that \texttt{toBytesBigEndian} is
equal to a \texttt{List.reverse ∘ toBytes'_ax} where \texttt{toBytes'_ax} is a
dummy definition copied from the EvmYul model.

As future work, we coudl submit a PR to the EvmYul model repository to change
these \texttt{private} attributes.

\begin{theorem}[Axiom: toBytesBigEndian_rw]
\label{thm:toBytesBigEndian_rw}\lean{Axioms.toBytesBigEndian_rw}
\texttt{EvmYul.toBytesBigEndian n = (List.reverse ∘ toBytes'_ax) n}
\end{theorem}

\subsection{Opcode Semantics}

The most relevant theorem of this kind is the behavior of the \texttt{X}
function (defined in \ref{def:EVM.X}) when executing a single-opcode program and
having the program counter be 1.

% \begin{definition}[EvmYul: Function X]\label{def:X}\lean{EvmYul.EVM.X}\leanok
% Function $X$ defined in the yellow paper.
% \end{definition}

We then specify the result of running the $X$ function on a bad opcode.

\begin{theorem}[X_bad_pc]\label{thm:X_bad_pc}\lean{X_bad_pc}\uses{def:EVM.X}\leanok
Given
\begin{itemize}
\item a single-opcode program $P$,
\item a gas amount greater than 1,
\item the program counter set to 1,
\item the symbolic and well constrained remaining inputs of $X$.
\end{itemize}
Executing $X$ on $P$ with a fully symbolic state with the above conditions
is successful and returns an empty output with the following modifications to
the symbolic state:
\begin{itemize}
\item The parameter \texttt{returnData} is empty.
\item The \texttt{executionLength} is increased by one.
\end{itemize}
\end{theorem}
\begin{proof}
\leanok
Massaging with tactics the execution path that evaluates if the program counter
is out of bounds derives this result.
\end{proof}

\section{KEVM Interface}

We use this
\href{https://runtimeverification.github.io/evm-equivalence/docs/EvmEquivalence/Interfaces/FuncInterface.html}{interface}
to provide intermediate lemmas or better representations for the K-generated
Lean definitions.

This interface contains results about the following categories:
\begin{itemize}
\item Simple function behavior: results about simple functions like arithmetic
  or boolean operations.
\item Behavior of \texttt{«#sizeWordStack»}: Equating \texttt{«#sizeWordStack»}
  to \texttt{List.length}.
\item Behavior of K's \texttt{in_keys} function.
\item Behavior of \texttt{#inStorage} function.
\item Results that have to do with memory-related operations.
\item Bytes manipulation: ease the reasoning about `SortBytes` manipulation functions.
\end{itemize}

\section{Gas Interface}

KEVM models \textbf{all} EVM hardforks (schedules). So, when executing a program in KEVM, we
need to keep track of which hardfork (schedule) are we executing with.
Knowing the schedule we're executing with allows us to know the correct gas cost
of an opcode, and to know which rules apply to our schedule of choice.

There are two functions involved in this:
\begin{definition}[\texttt{_<_>}]\label{def:gasConstSched}\lean{«_<_>_SCHEDULE_Int_ScheduleConst_Schedule»}\leanok
The \texttt{«_<_>_SCHEDULE_Int_ScheduleConst_Schedule»} function (or \texttt{_<_>} for
short) takes as arguments a gas constant (such as $G_{\text{verylow}}$) and a
schedule (such as CANCUN) and returns how much is that gas constant in that
schedule.
\end{definition}
\begin{definition}[\texttt{_<<_>>}]\label{def:flagConstSched}\lean{«_<<_>>_SCHEDULE_Bool_ScheduleFlag_Schedule»}\leanok
The function \texttt{«_<<_>>_SCHEDULE_Bool_ScheduleFlag_Schedule»} (or
\texttt{_<<_>>} for short) takes as arguments a schedule flag (such as
$G_{\text{hascreate2}}$) and  a schedule and returns the corresponding boolean
if the schedule has the schedule flag. This is used to discriminate which rules
can be applied depending on the schedule we're working with.

In this example, working on an schedule \texttt{SCHED}, any rule involving the
\texttt{CREATE2} opcode will have to satisfy $G_{\text{hascreate2}}$\texttt{<<
  SCHED >>} in order to be applied.
\end{definition}

Given that EvmYul works with the Cancun schedule, the two main results of this
\href{https://runtimeverification.github.io/evm-equivalence/docs/EvmEquivalence/Interfaces/GasInterface.html}{interface}
are a plain rewrite of the \texttt{_<_>} and \texttt{_<<_>>} functions to the
Cancun schedule.

\begin{theorem}[cancun_def]
\label{thm:cancun_def}\lean{GasInterface.cancun_def}\leanok\uses{def:gasConstSched}
Given \texttt{const : SortScheduleConst}, we equate
\texttt{«_<_>_SCHEDULE_Int_ScheduleConst_Schedule» const .CANCUN_EVM} to a
simple match statement on \texttt{const}.
\end{theorem}
\begin{proof}
\leanok
This result follows from applying the custom tactics found in
\href{https://runtimeverification.github.io/evm-equivalence/docs/EvmEquivalence/Interfaces/Tactics.html}{\texttt{Tactics.lean}},
which boil down to a bunch of simplifications.
\end{proof}

\begin{theorem}[flag_cancun_def]
\label{thm:flag_cancun_def}\lean{GasInterface.flag_cancun_def}\leanok\uses{def:flagConstSched}
Given \texttt{flag : SortScheduleFlag}, we equate
\texttt{«_<<_>>_SCHEDULE_Bool_ScheduleFlag_Schedule» flag .CANCUN_EVM} to a
simple match statement on \texttt{flag}.
\end{theorem}
\begin{proof}
\leanok
This result follows from applying the custom tactics found in
\href{https://runtimeverification.github.io/evm-equivalence/docs/EvmEquivalence/Interfaces/Tactics.html}{\texttt{Tactics.lean}},
which boil down to a bunch of simplifications.
\end{proof}
