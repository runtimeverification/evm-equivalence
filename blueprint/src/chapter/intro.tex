\chapter*{Evm Models Equivalence}\label{evm-equiv-intro}

The project's goal is to prove the equivalence between two mathematical models of the EVM.
The mathematical models are Nethermind's \href{https://github.com/nethermindEth/EVMYulLean/}{EvmYul}
and Runtime Verification's \href{https://github.com/runtimeverification/evm-semantics}{KEVM}.

Notably, these two models are written in very different languages: EvmYul in
\href{https://lean-lang.org}{Lean 4} and KEVM in \href{https://kframework.org}{K}.
The first step to prove equivalence is to have a framework in which to compare
what the models do, and then show the equivalence of it.

To this end, the K framework allows for the generation of Lean 4 code
representing compiled K definitions.
Hence, the methodology is to show in Lean 4 the equivalence between EvmYul and
the generated Lean 4 code from the KEVM definitions.

This blueprint will (for the time being) only concern about the Lean proofs on
equivalence. For documentation on the more engineering-related topics please
refer to the
\href{https://github.com/runtimeverification/evm-equivalence}{repository}.

\section*{Overall Approach}

What follows is a summary of what is explained in the blueprint.

This project consists of several parts that lead up to proving equiavlence
between the EvmYul model and KEVM.

\subsubsection*{Code generation}

The most fundamental layer from the KEVM side is to extract the compiled KEVM
definitions to Lean 4. We do this by means of the \texttt{klean} tool.

There are three things of special importance to the generated code.

\begin{itemize}
\item \textbf{Function termination:} There are some generated functions that
  Lean can't automatically deduce its termination argument. Modify the code to
  provide it.
\item \textbf{Uninterpreted functions:} Some functions require an ad-hoc
  implementation by the Lean 4 backend. When this implementation is not provided
  they are declared as axioms.
\item \textbf{\texttt{Rewrites} type:} The \texttt{Rewrites} is formatted for
  better reading.
\end{itemize}

Refer to to \ref{chap:gencode} for details.

\subsubsection*{Reasoning Interfaces}

The second layer is that of reasoning interfaces. Files containing results that
make reasoning about both models more ammenable.

The results present in these files cover from basic facts about the types we're
dealing with to conversions to more friendly representations of convoluted
expressions.

KEVM's \texttt{GasInterface.lean} is particulary time-intensive to execute. For
debugging anything unrelated to it it's advised to \texttt{sorry} our its
proofs.

See \ref{chap:interfaces} for more details.

\subsubsection*{EvmYul Summaries}

There is a series of files containing results that encapsulate the computational
behavior of EvmYul's functions for specific opcodes or classes of opcodes. This
allows us to have a precise description of the state updates that each opcode
entails.

Note that we also have the equivalent of these summaries for KEVM. However,
they're produced differntly: KEVM computes the states updates and then they are
exported as the constructors of the \texttt{Rewrites} type.

For a more detailed discussion see \ref{chap:EvmYulSummaries}.

\subsubsection*{State Mapping}

Once we have reasonable reasoning interfaces and the summaries of the
computational behavior at the opcode level for each model, we define a function
that maps states from one model to the other.

So far, we have a mapping from KEVM states to EvmYul ones. It remains as future
work to do the inverse direction.

For a more detailed discussion see \ref{chap:statemap}.

\subsubsection*{Proving Equivalence}

The last step is to prove equivalence for each opcode. For a formal deinition of
the equivalence definition and proof strategy see \ref{chap:equivalences}.

\subsubsection*{Axioms and sorries}

We employ axioms for different uses, from generated uninterpreted functions to
blueprints for future function implementations or theorems that should be proven
at some point.

See \ref{chap:axioms} for a more detailed description.

