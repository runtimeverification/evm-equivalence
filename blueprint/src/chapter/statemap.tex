\chapter{State Map}

One of the key components of this project is the notion of state mapping. Since
we want to show that KEVM and the EvmYul models agree on opcode execution, we
need a notion of state equivalence to evaluate the agreement.

So far we are only doing one side of the equiavlence. Namely, from KEVM to
EvmYul. This shows that what KEVM does is agreed on by the EvmYul model. Hence,
the state mapping we present here will map a KEVM state into an EvmYul one.

\section{Type Mapping}

The first thing for mapping KEVM states to EvmYul states is to convert KEVM
types to their corresponding EvmYul ones.
Because KEVM operates on integers and then imposes restrictions on their value
when applying rules, such restrictions don't exist at the type level. This is
important to keep in mind since when mapping integers representing EVM values,
we don't have to translate the full generality of the KEVM types. But we need to
be mindful on how the integers are used in each instance in order to provide an
accurate mapping.

The first step is to choose a default on how to map integers themselves to
EvmYul's \texttt{UInt256}.

\begin{definition}[intMap]\lean{StateMap.intMap}\leanok\label{def:intMap}
Given $n$ of type \texttt{SortInt},
$\text{intMap}\, n =\,
$\href{https://runtimeverification.github.io/evm-equivalence/docs/EvmYul/UInt256.html#EvmYul.UInt256.toSigned}{UInt256.toSigned}$\,n$.
\end{definition}

Once we have the mapping of one of the fundamental KEVM underlying types, we
have to map KEVM's cells. K cells are extracted to Lean as \texttt{struct}s that
act as a wrapper for an integer value or for another wrapper.

As an example, take \texttt{SortGasCell},  Lean's extraction of KEVM's gas cell
which is used to track the amount of gas left:

\begin{verbatim}
structure SortGasCell : Type where
  val : SortGas
\end{verbatim}

We see that it's a wrapper for a \texttt{SortGas} value, and that
\texttt{SortGas} follows the same structure:

\begin{verbatim}
inductive SortGas : Type where
  | inj_SortInt (x : SortInt) : SortGas
\end{verbatim}

The fact that \texttt{SortGasCell} has a \texttt{SortGas} value and not a plain
\texttt{SortInt} is to avoid the mixing of integer values at the cell level.

Given this, the mapping of \texttt{SortGasCell} will come down to unwrapping it
to its \texttt{SortGas} value and in turn map its \texttt{SortInt} contents
through the \texttt{intMap} function to EvmYul's \texttt{UInt256}.

\section{State Mapping}

Once we have mapped a subset of KEVM generated types to their corresponding
EvmYul's images, we can map KEVM states to EvmYul ones.

\begin{definition}[stateMap]
\lean{StateMap.stateMap}\leanok\label{def:stateMap}\uses{def:intMap}

KEVM states are represented by the type
\href{https://runtimeverification.github.io/evm-equivalence/docs/EvmEquivalence/KEVM2Lean/Sorts.html#SortGeneratedTopCell}{\texttt{SortGeneratedTopCell}}.
Hence, we will be taking as input a \texttt{SortGeneratedTopCell} and converting
it to an
\href{https://runtimeverification.github.io/evm-equivalence/docs/EvmYul/EVM/State.html#EvmYul.EVM.State}{\texttt{EVM.State}}.

The state mapping function will take an extra \texttt{EVM.State} argument to
perform the identity mapping on the parts of the state that are not yet mapped.
\end{definition}

\section{Axiomatic Mapping}

Some of the state mapping is done in an axiomatic manner. This is because of two
different reasons. First, have not yet committed to a definitive map structure
for the Lean extracted code. Second, since the EvmYul model is more geared
towards execution than reasoning, we've decided to not commit to their
underlying map structure based on Red-Black trees.

To this effect, we postulate as axioms the existence of the
\href{https://runtimeverification.github.io/evm-equivalence/docs/EvmEquivalence/Interfaces/Axioms.html}{following
  functions}:

\begin{itemize}
\item \texttt{SortAccountsCellMap}: maps \texttt{SortAccoutsCell} to
  \texttt{AccountMap}.
\item \texttt{SortAccessedStorageCellMap}: maps \texttt{SortAccessedStorageCell}
  to the type of \texttt{accessedStorageKeys}.
\item \texttt{SortStorageCellMap}: maps \texttt{SortStorageCell} to
  \texttt{Storage}.
\item \texttt{SortTransientStorageCellMap}: maps
  \texttt{SortTransientStorageCell} to \texttt{Storage}.
\end{itemize}

\subsection{Axioms as pre-theorems}

Since these are uninterpreted functions, we also assert
\href{https://runtimeverification.github.io/evm-equivalence/docs/EvmEquivalence/StateMap.html#Axioms.findAccountInAccountCellMap}{behavioral
  axioms} on how these functions have to behave. These axioms serve as a
blueprint for any future implementation of these functions, since whatever
implementation is chosen should render the behavioral axioms as theorems.
