\chapter{EvmYul Summaries}\label{chap:EvmYulSummaries}

A prior step to proving any notion of equivalence is to encapsulate or summarize
the computational content of both models. In the KEVM case, the computational
content is summarized by a rewrite rule on the KEVM side, and then extracted to
Lean as shown in the
\href{https://runtimeverification.github.io/evm-equivalence/docs/EvmEquivalence/KEVM2Lean/Rewrite.html}{Rewrite.lean}
file.

The way summarizing the computational content for the EvmYul model will come in
the form of theorems. To know which theorems we need to take a step back and
look at which parts of the two models we want to equate. On the KEVM side, it's
the summaries in form of constructors of the \texttt{Rewrites} type.
On the EvmYul side, we don't have as fine-grained semantic steps as KEVM's
rewrite rules for opcode execution, so we will produce theorems that state the
specific semantic steps we're interested in. With this in
mind, we will focus on two of the core EvmYul functions: \texttt{EVM.step} and
\texttt{EVM.X}.

\begin{definition}[EVM.step]
\label{def:EVM.step}\lean{EvmYul.EVM.step}\leanok
\texttt{EVM.step} orchestrates opcode execution, taking the following arguments.
\begin{itemize}
\item Amount of gas left: \texttt{fuel}
\item Cost of executing the opcode: \texttt{gasCost}
\item Opcode to be executed: \texttt{instr}
\end{itemize}

Given these arguments and an \texttt{EVM.state}, the \texttt{step} function will
perform the correspondent updates on the provided state according to the
semantics of \texttt{instr}.
\end{definition}

From this definition there are a couple of things to be noted.
\begin{itemize}
\item The \texttt{instr} parameter is of type \texttt{Option} and as such
can be \texttt{none}, in which case the opcode is fetched from the execution
environment's bytecode.
\item \texttt{gasCost} is always given by functions calling \texttt{step}.
\item If \texttt{fuel} is not strictly positive, \texttt{step} will immediately fail.
\end{itemize}
This means that while the \texttt{step} function captures a big portion of the
executed opcode's semantics, it is not the full picture.

To incorporate a most complete picture of opcode semantics execution we turn to
the \texttt{EVM.X} function. This function iterates the above \texttt{EVM.step}
and performs several important actions.

\begin{definition}[EVM.X]
\label{def:EVM.X}\lean{EvmYul.EVM.X}\leanok\uses{def:EVM.step}
Formalization of the yellow paper's $X$ function as defined in section 9.4.

Given the following arguments:
\begin{itemize}
\item Amount of gas left: \texttt{fuel}.
\item Valid jumpdests: \texttt{validJumps}.
\item An \texttt{EVM.State}: \texttt{evmState}.
\end{itemize}

The function \texttt{X} performs the following tasks:
\begin{itemize}
\item Decode the opcodes present in \texttt{evmState}'s execution
environment.
\item Perform sanity checks and halt execution if these are not met.
\item Assign gas costs depending on the decoded opcode.
\item Execute \texttt{EVM.step} with the current decoded opcode.
\item Iterate if the conditions are right.
\end{itemize}
\end{definition}

Since the \texttt{X} function operates on the \texttt{evmState}'s bytecode, in
order to exercise its semantic content we will execute it but restricting
\texttt{evmState}'s bytecode to be a one-opcode program.

\section{\texttt{EVM.step} and \texttt{EVM.X} summary}

We can now state summarization theorems for \texttt{step} and \texttt{X}. Given
a fully symbolic \texttt{EVM.State} $S$, summarization theorems clearly state
which are the precise state updates that occur to $S$ after executing the
\texttt{step} or \texttt{X} functions with the following:

\begin{itemize}
\item A set of concrete opcodes.
\item Enough gas to perform execution.
\item Well-formedness of other inputs to the functions such as a symbolic stack.
\end{itemize}

The sets of concrete opcodes are created by grouping opcodes that have similar
effects to $S$.
For instance, we can
\href{https://runtimeverification.github.io/evm-equivalence/docs/EvmEquivalence/Summaries/StackOperationsSummary.html}{group
  all opcodes} that perform operations on the stack such as \texttt{ADD},
\texttt{SHL} and so on, and have a general enough theorem that captures the
effects of these opcodes on $S$ without having to do a separate summary for each
of them.
